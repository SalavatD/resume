\documentclass{resume}


\begin{document}


% Оглавление
\begin{tabular*}{\textwidth}{l@{\extracolsep{\fill}}r}
\textbf{\LargeСалават Даутов}&\href{tel:+79378353046}{+7-937-835-3046} \\
\href{https://github.com/SalavatD}{\color{blue}\textbf{Мой GitHub}}&\href{mailto:dautovsalavatd@gmail.com}{dautovsalavatd@gmail.com} \\

\end{tabular*}


% Образование
\section{Образование}
\resumeSubHeadingListStart
\resumeSubheading{Национальный исследовательский университет <<МИЭТ>>}{Зеленоград, Москва}
{Информатика и вычислительная техника, 3 курс, бакалавриат}{Сентябрь 2018 -- настоящее время}
\resumeSubHeadingListEnd


% Навыки программирования
\section{Навыки программирования}
\resumeSubHeadingListStart
\item{\textbf{Основные языки:} Java, C++.}
\item{\textbf{Также имею опыт работы с:} Python, C, C\#, SQL.}
\resumeSubHeadingListEnd


% Технические навыки
\section{Технические навыки}
\resumeSubHeadingListStart
\item{Имею опыт администрирования  серверов на базе \textbf{Linux}.}
\item{Знаком с технологиями \textbf{JDBC} и \textbf{Hibernate}.}
\item{Работал с БД \textbf{MySQL} и \textbf{MS SQL} и веб-серверами \textbf{Apache} и \textbf{nginx}.}
\item{Большой опыт использования \textbf{git}.}
\item{Уровень владения английским \textbf{Pre-Intermediate}.}
\resumeSubHeadingListEnd


% Проекты
\section{Проекты}
\resumeSubHeadingListStart
\item{\textbf{Приложение для определения стоимости товаров в продуктовой корзине} \\
Это был проект для хакатона. Мы сделали простое приложение, которое определяет стоимость товаров по фотографии продуктовой корзины. Наша команда заняла второе место. Я занимался разработкой прототипа клиента для смартфона. \\\textbf{Используемые технологии:} Для распознавания объектов на фотографии использовался сервис \textbf{Azure Custom Vision}, клиент был написан на \textbf{Xamarin}. \\
\href{https://github.com/SalavatD/MaiCsHackathon}{\color{blue}Ссылка на репозиторий}.}

\item{\textbf{MSP Game Hack} \\
Наша команда за два дня разработала трехуровневую игру на определенную тему. Каждый участник команды разрабатывал один из уровней игры. Мы взяли приз зрительских симпатий. \\
\textbf{Используемые технологии:} Для создания использовался игровой движок \textbf{Unity}.}

\item{\textbf{MoscowHackApplication} \\
Проект был разработан на Moscow.Hack 2019. Перед нами стояла задача: оптимизировать работу курьеров. Мы должны были разработать приложение, которое позволяло бы курьерам за кратчайший пройденный путь доставлять наибольшее число посылок. Я занимался разработкой клиента для курьеров. \\
\textbf{Используемые технологии:} Клиент был написан с использованием \textbf{Xamarin}. \\
\href{https://github.com/SalavatD/MoscowHackApplication}{\color{blue}Ссылка на репозиторий}.}

\item{\textbf{Snake Online} \\
Клиент-серверная реализация игры змейка. \\
\textbf{Используемые технологии:} Сервер был написан с использованием \textbf{ASP.NET Core}, клиент реализован на \textbf{WPF}. \\
\href{https://github.com/SalavatD/Snake}{\color{blue}Ссылка на репозиторий}.}

\item{\textbf{Wieno game} \\
Приложение, которое я разработал во время изучения Unity. \\
\textbf{Используемые технологии}: Игровой движок \textbf{Unity}. \\
\href{https://github.com/SalavatD/Wieno}{\color{blue}Ссылка на репозиторий}. \href{https://play.google.com/store/apps/details?id=com.anytrash.wieno}{\color{blue}Ссылка на Google Play}.}
\resumeSubHeadingListEnd


% Стажировки и курсы
\section{Стажировки и курсы}
\resumeSubHeadingListStart
\item{\textbf{Kaspersky SafeBoard 2019} \\
Обучался на стриме разработки в Лаборатории Касперского. За время обучения написал клиент-серверное Web API приложение. \\
\textbf{Используемые технологии:} Для создания сервера использовался \textbf{ASP.NET Core}.}
\resumeSubHeadingListEnd


% Онлайн курсы
\section{Онлайн курсы}
\resumeSubHeadingListStart
\item{\textbf{Искусство разработки на современном C++} \\
Прошел 4 из 5 курсов специализации по разработке на C++ 17 от МФТИ и Яндекса на платформе Coursera. \\
\href{https://github.com/SalavatD/SalavatD.github.io/tree/master/\%D0\%98\%D1\%81\%D0\%BA\%D1\%83\%D1\%81\%D1\%81\%D1\%82\%D0\%B2\%D0\%BE\%20\%D1\%80\%D0\%B0\%D0\%B7\%D1\%80\%D0\%B0\%D0\%B1\%D0\%BE\%D1\%82\%D0\%BA\%D0\%B8\%20\%D0\%BD\%D0\%B0\%20\%D1\%81\%D0\%BE\%D0\%B2\%D1\%80\%D0\%B5\%D0\%BC\%D0\%B5\%D0\%BD\%D0\%BD\%D0\%BE\%D0\%BC\%20C\%2B\%2B}{\color{blue}Ссылка на сертификаты}.}
\resumeSubHeadingListEnd

\end{document}
